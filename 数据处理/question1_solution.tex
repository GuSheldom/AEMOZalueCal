\section*{Question 1 - Part 2: 求解线性方程组}

\subsection*{问题陈述}
给定矩阵 $\mathbf{A}$ 和向量 $\mathbf{b}$:
$$\mathbf{A} = \begin{bmatrix}
0 & 3 & 0 & 0 & 0 \\
-2 & 0 & 1 & 0 & 0 \\
0 & -1 & 0 & -1 & 0 \\
0 & 0 & -1 & 0 & 4 \\
0 & 0 & 0 & 1 & 0
\end{bmatrix}, \quad \mathbf{b} = \begin{bmatrix} b_1 \\ b_2 \\ b_3 \\ b_4 \\ b_5 \end{bmatrix}$$

求解线性方程组 $\mathbf{A}\mathbf{x} = \mathbf{b}$ 的解集 $\{\mathbf{x} : \mathbf{A}\mathbf{x} = \mathbf{b}\}$。

\subsection*{解答步骤}

\textbf{步骤1: 矩阵分析}

首先分析矩阵 $\mathbf{A}$ 的基本性质:
- 矩阵维度:$5 \times 5$
- 通过计算可得:$\text{rank}(\mathbf{A}) = 4$(不是满秩)
- 自由变量个数:$5 - 4 = 1$

\textbf{步骤2: 行简化}

对增广矩阵 $[\mathbf{A}|\mathbf{b}]$ 进行行简化得到:

$$\text{RREF}[\mathbf{A}|\mathbf{b}] = \begin{bmatrix}
1 & 0 & 0 & 0 & -2 & 0 \\
0 & 1 & 0 & 0 & 0 & 0 \\
0 & 0 & 1 & 0 & -4 & 0 \\
0 & 0 & 0 & 1 & 0 & 0 \\
0 & 0 & 0 & 0 & 0 & 1
\end{bmatrix}$$

\textbf{步骤3: 解的存在性分析}

从行简化结果的最后一行 $[0 \quad 0 \quad 0 \quad 0 \quad 0 \quad | \quad 1]$ 可以看出:

$$0 = 1$$

这意味着方程组 $\mathbf{A}\mathbf{x} = \mathbf{b}$ 只有在特定条件下才有解。

通过分析原始的行简化过程,我们发现解存在的必要充分条件是:
$$\boxed{b_5 = 0}$$

\textbf{步骤4: 当 $b_5 = 0$ 时的解}

当 $b_5 = 0$ 时,方程组变为:
$$\begin{bmatrix}
0 & 3 & 0 & 0 & 0 \\
-2 & 0 & 1 & 0 & 0 \\
0 & -1 & 0 & -1 & 0 \\
0 & 0 & -1 & 0 & 4 \\
0 & 0 & 0 & 1 & 0
\end{bmatrix} \begin{bmatrix} x_1 \\ x_2 \\ x_3 \\ x_4 \\ x_5 \end{bmatrix} = \begin{bmatrix} b_1 \\ b_2 \\ b_3 \\ b_4 \\ 0 \end{bmatrix}$$

从行简化结果可得:
\begin{align}
x_1 - 2x_5 &= 0 \quad \Rightarrow \quad x_1 = 2x_5 \\
x_2 &= 0 \\
x_3 - 4x_5 &= 0 \quad \Rightarrow \quad x_3 = 4x_5 \\
x_4 &= 0
\end{align}

其中 $x_5$ 是自由变量。

\textbf{步骤5: 零空间}

齐次方程组 $\mathbf{A}\mathbf{x} = \mathbf{0}$ 的解(零空间)为:
$$\text{Null}(\mathbf{A}) = \text{span}\left\{\begin{bmatrix} 2 \\ 0 \\ 4 \\ 0 \\ 1 \end{bmatrix}\right\}$$

\subsection*{最终答案}

线性方程组 $\mathbf{A}\mathbf{x} = \mathbf{b}$ 的解集为:

$$\boxed{
\{\mathbf{x} : \mathbf{A}\mathbf{x} = \mathbf{b}\} = 
\begin{cases}
\emptyset & \text{如果 } b_5 \neq 0 \\
\left\{t\begin{bmatrix} 2 \\ 0 \\ 4 \\ 0 \\ 1 \end{bmatrix} : t \in \mathbb{R}\right\} & \text{如果 } b_5 = 0
\end{cases}
}$$

\subsection*{解释}

1. **当 $b_5 \neq 0$ 时**:方程组无解,因为增广矩阵的秩大于系数矩阵的秩。

2. **当 $b_5 = 0$ 时**:方程组有无穷多解,解集是一条通过原点的直线,方向向量为 $\begin{bmatrix} 2 & 0 & 4 & 0 & 1 \end{bmatrix}^T$。

这是因为矩阵 $\mathbf{A}$ 不可逆(行列式为0),存在非平凡的零空间。

