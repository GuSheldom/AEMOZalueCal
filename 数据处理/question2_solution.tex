\documentclass[12pt]{article}
\usepackage[utf8]{inputenc}
\usepackage{amsmath}
\usepackage{amsfonts}
\usepackage{amssymb}
\usepackage{geometry}

\geometry{a4paper, margin=1in}

\title{Question 2: Rock n Roll Die Solution}
\author{}
\date{}

\begin{document}

\maketitle

\noindent \textbf{Given:} A four-faced die with equal probabilities. Two faces have value 2, one face has value 3, and one face has value 4.

\section*{(a) Probabilities of getting face values 2, 3, and 4 [1 mark]}

Since each face has equal probability $\frac{1}{4}$:

\begin{align}
P(\text{face value} = 2) &= \frac{2}{4} = \frac{1}{2} \\
P(\text{face value} = 3) &= \frac{1}{4} \\
P(\text{face value} = 4) &= \frac{1}{4}
\end{align}

Verification: $\frac{1}{2} + \frac{1}{4} + \frac{1}{4} = \frac{2}{4} + \frac{1}{4} + \frac{1}{4} = 1$ ✓

\section*{(b) Probability that sum of two independent rolls is 6 [3 marks]}

For the sum to be 6, we need the following combinations:
\begin{itemize}
\item $(2, 4)$: $P = \frac{1}{2} \times \frac{1}{4} = \frac{1}{8}$
\item $(3, 3)$: $P = \frac{1}{4} \times \frac{1}{4} = \frac{1}{16}$
\item $(4, 2)$: $P = \frac{1}{4} \times \frac{1}{2} = \frac{1}{8}$
\end{itemize}

Note that $(2, 4)$ and $(4, 2)$ each occur in multiple ways due to the two faces with value 2.

Total probability:
\begin{align}
P(\text{sum} = 6) &= 2 \times \frac{1}{8} + \frac{1}{16} + 2 \times \frac{1}{8} \\
&= \frac{1}{4} + \frac{1}{16} + \frac{1}{4} \\
&= \frac{4}{16} + \frac{1}{16} + \frac{4}{16} \\
&= \frac{9}{16}
\end{align}

\section*{(c) Expected value and variance of product X [5 marks]}

Let $Y$ denote a single roll. First, compute $E[Y]$ and $\text{Var}[Y]$:

\begin{align}
E[Y] &= 2 \times \frac{1}{2} + 3 \times \frac{1}{4} + 4 \times \frac{1}{4} \\
&= 1 + \frac{3}{4} + 1 = \frac{11}{4}
\end{align}

\begin{align}
E[Y^2] &= 4 \times \frac{1}{2} + 9 \times \frac{1}{4} + 16 \times \frac{1}{4} \\
&= 2 + \frac{9}{4} + 4 = \frac{8 + 9 + 16}{4} = \frac{33}{4}
\end{align}

\begin{align}
\text{Var}[Y] &= E[Y^2] - (E[Y])^2 \\
&= \frac{33}{4} - \left(\frac{11}{4}\right)^2 \\
&= \frac{33}{4} - \frac{121}{16} \\
&= \frac{132 - 121}{16} = \frac{11}{16}
\end{align}

For the product $X = Y_1 \times Y_2$ where $Y_1$ and $Y_2$ are independent:

\begin{align}
E[X] &= E[Y_1 \times Y_2] = E[Y_1] \times E[Y_2] \\
&= \frac{11}{4} \times \frac{11}{4} = \frac{121}{16}
\end{align}

For the variance, using the formula for independent random variables:
\begin{align}
\text{Var}[X] &= E[Y_1]^2 \text{Var}[Y_2] + E[Y_2]^2 \text{Var}[Y_1] + \text{Var}[Y_1] \text{Var}[Y_2] \\
&= \left(\frac{11}{4}\right)^2 \times \frac{11}{16} + \left(\frac{11}{4}\right)^2 \times \frac{11}{16} + \frac{11}{16} \times \frac{11}{16} \\
&= \frac{121}{16} \times \frac{11}{16} + \frac{121}{16} \times \frac{11}{16} + \frac{121}{256} \\
&= 2 \times \frac{1331}{256} + \frac{121}{256} \\
&= \frac{2662 + 121}{256} = \frac{2783}{256}
\end{align}

\section*{(d) Probability that three independent rolls are all different [3 marks]}

For three face values to be all different, we need exactly one 2, one 3, and one 4.

The number of arrangements is $3! = 6$: $(2,3,4)$, $(2,4,3)$, $(3,2,4)$, $(3,4,2)$, $(4,2,3)$, $(4,3,2)$.

Each arrangement has probability:
\begin{align}
P(\text{one arrangement}) = \frac{1}{2} \times \frac{1}{4} \times \frac{1}{4} = \frac{1}{32}
\end{align}

Total probability:
\begin{align}
P(\text{all different}) = 6 \times \frac{1}{32} = \frac{6}{32} = \frac{3}{16}
\end{align}

\section*{(e) Expected reward E[R] [4 marks]}

Following the hint, let $R_j$ denote the expected reward from now on, assuming the previous roll was $j$.

Setting up the equations:
\begin{align}
R_2 &= 0.5 \times 2 + 0.25 \times (3 + R_3) + 0.25 \times (4 + R_4) \\
R_3 &= 0.5 \times (2 + R_2) + 0.25 \times 3 + 0.25 \times (4 + R_4) \\
R_4 &= 0.5 \times (2 + R_2) + 0.25 \times (3 + R_3) + 0.25 \times 4
\end{align}

Simplifying:
\begin{align}
R_2 &= 1.75 + 0.25R_3 + 0.25R_4 \\
R_3 &= 1.75 + 0.5R_2 + 0.25R_4 \\
R_4 &= 2 + 0.5R_2 + 0.25R_3
\end{align}

Solving this system of equations, we get:
\begin{align}
R_2 &= \frac{9}{2} \\
R_3 &= \frac{27}{5} \\
R_4 &= \frac{28}{5}
\end{align}

The expected reward at the start of the game:
\begin{align}
E[R] &= \frac{1}{2} \times \left(2 + \frac{9}{2}\right) + \frac{1}{4} \times \left(3 + \frac{27}{5}\right) + \frac{1}{4} \times \left(4 + \frac{28}{5}\right) \\
&= \frac{1}{2} \times \frac{13}{2} + \frac{1}{4} \times \frac{42}{5} + \frac{1}{4} \times \frac{48}{5} \\
&= \frac{13}{4} + \frac{42}{20} + \frac{48}{20} \\
&= \frac{13}{4} + \frac{90}{20} \\
&= \frac{13}{4} + \frac{9}{2} \\
&= \frac{13 + 18}{4} = \frac{31}{4}
\end{align}

\section*{Final Answers}

\begin{itemize}
\item[(a)] $P(2) = \frac{1}{2}$, $P(3) = \frac{1}{4}$, $P(4) = \frac{1}{4}$
\item[(b)] $P(\text{sum} = 6) = \frac{9}{16}$
\item[(c)] $E[X] = \frac{121}{16}$, $\text{Var}[X] = \frac{2783}{256}$
\item[(d)] $P(\text{all different}) = \frac{3}{16}$
\item[(e)] $E[R] = \frac{31}{4}$
\end{itemize}

\end{document}
